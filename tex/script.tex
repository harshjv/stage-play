\documentclass[12pt,a4paper,oneside]{memoir}
\usepackage[english]{babel}
\usepackage[utf8]{inputenc}
\usepackage{enumitem}

\newlength{\drop}
\chapterstyle{default}
\pagestyle{myheadings}
\setlength{\parindent}{0pt}

\renewcommand{\printtoctitle}[1]{\centering\Large\bfseries Acts}
\pagenumbering{gobble}

\newcommand*{\titleGM}{\begingroup
   \drop = 0.1\textheight
   \vspace*{\baselineskip}
   \vfill
   \hbox{
       \hspace*{0.2\textwidth}
       \rule{1pt}{\textheight}
       \hspace*{0.05\textwidth}
       \parbox[b]{0.75\textwidth}{
           \vbox{
               \vspace{\drop}{\noindent\HUGE\bfseries Paisa Bolta Hai}\\
               [2\baselineskip]{\huge\itshape Humorous Nukkad Natak}\\
               [4\baselineskip]{\Large The Nautankis}\par\vspace{0.5\textheight}
               {\noindent \textbf{Verve} \\[0.5\baselineskip] \textbf{2015}}\\
               [\baselineskip]
           }
       }
   }
   \vfill
   \null
\endgroup}

\begin{document}

\titleGM

\pagenumbering{roman}
\tableofcontents*
\clearpage

\pagenumbering{arabic}

\markright{\textsc{Paisa Bolta Hai}}

%%%%%%%%%%%%%%%%
\chapter*{Entry}
%%%%%%%%%%%%%%%%
\addcontentsline{toc}{chapter}{Entry}

\begin{description}[itemsep=1ex,leftmargin=1cm]

\item[] \hfill \\
\textit{Enter from the left side of the stage forming a half ellipse then towards a straight line.}

\item[RIYA] \textit{(calls)} Suno suno sab suno suno.

\item[ALL] Kaho kaho kya kehna hai?

\item[RIYA] \textit{(again)} Aree, suno suno sab suno suno.

\item[ALL] Abb Bol bhi do kya kehna hai?

\item[JAINAM] Ae madaniya ke lala, bitiya kuch kehna chahat hai, tanik dhyan toh do.
\item[VIJAAN] Bolo bitiya kya batiyana hai? Umm \textit{(thinking)} Tumko kya koi paiso ki jarurat?

\item[SALONI] Paiso ki jarurat toh Kuber aur Laxmi ko chodke aaj sabko hi hai.

\item[NEHA] Sach hi keh rahi hai ye bitiya.

\item[KAVITA] Bin jarurat lalchiyo ki paise mangane ki aadat,
          Aab bass \textit{(pause)} yahi reh gyi hai, desh ki sabse badi aafat.

\item[ALL] \textit{(loudly)} Bin jarurat lalchiyo ki paise mangane ki aadat,
          Aab bass \textit{(pause)} yahi reh gyi hai, desh ki sabse badi aafat.

\item[HARSH] \textit{(walking towards audience)} Are bhai, aaj kal logo ki soch esi hai ki, paisa hai toh sab kuch hai, warna aadmi tucch hai.

\item[ALL] \textit{(everyone in different style)} Tucch.. Tucch.. Tucch..

\item[KAVITA] \textit{(stepping ahead)} Wahi toh! Jaha dekho waha paiso ka bolbala hai. Nind me bhi aaj kal logo ko sirf paisa hi dikhta hai.

\item[BHARGAV] \textit{(Act while sleeping, day dreaming)} Paisa, ruppaiya. Baadi gaadi, bunglow.. Paisa..

\item[KAVITA] Yahi toh aaj ka insaan sochta hai, Yahi ke \textbf{\textit{Paisa Bolta Hai}}
\item[ALL] Ha bhai \textbf{\textit{Paisa Bolta Hai}}

\end{description}

%%%%%%%%%%%%%%%%
\chapter*{Exam}
%%%%%%%%%%%%%%%%
\addcontentsline{toc}{chapter}{Exam}

\begin{center}


\textbf{Characters}

\vskip 1cm

\textbf{Riya Talaviya}, Student \\
\textbf{Saloni Bhagat}, Student \\
\textbf{Vijaan Masani}, Gunda \\
\vskip 1cm
\end{center}

\begin{description}[itemsep=1ex,leftmargin=1cm]
\setlength{\parskip}{5pt}

\item[] \hfill \\
\textit{Two girls are reading a book and a boy comes and starts bulling.}

\item[VIJAAN] \textit{(singing)} Khudko kya samajti hai, kitna akadti hai. College me nayi nayi aayi ek ladki hai.

\item[RIYA] \textit{(no reactions)}

\textit{Boy goes to the second girl starts teasing her.}

\item[VIJAAN] Aare, dekho iss maadam ko. Aakkad ki paribhasha toh jaha ye khadi ho jaaye, wahi se shuru hoti hai. \textit{(laughs)} Ha ha ha.

\item[SALONI] (aggressively) Oyee gunday! Pehle khudko dekh baadme kisiko kuch kehna. Exam hai. Padhne de. Tera kya? Tu bass gundagiri kar, aur baadme exam me fail. \textit{(laughs)} Ha ha ha.

\item[RIYA] (pulling SALONI over) Jaane de, jaane de, kyu panga le rahi hai.

\item[VIJAAN] (boasting) Tu janti nahi hai me kon hu?

\item[SALONI] Tu kon hai, kya hai wo toh result wale din milke bolna.

\item[VIJAAN] \textit{(laughs)} Ha ha ha. Dekh lenge.

\item[BHARGAV] \textit{(seriously)} Result ka din! (carelessly) Oyee, mera result to dekh.

\item[ALL] \textit{(heading over the result board)}

\item[SALONI] \textit{(looks at the result, crying)} Ohh no! Me fail kese ho gayi. Exams toh ekdam mast gayi thi.

\item[VIJAAN] (comes in center, pushes everyone aside) Hato sab. Mujhe result dekhne do.

\textit{(after watching result)} 

\item[VIJAAN] Areey wah! (looking at SALONI, boasting) First class se pass madam. First class.

\item[SALONI] \textit{(exclaimed)} Tum paas aur me fail? Ye kese ho gaya?

\item[VIJAAN] \textit{(sings and concludes)} \\
Na boyfriend na baccha, \\
Na baap bada na bhaiya, \\
The whole thing is that k bhaiya, sabse baada rupaiya!

\end{description}
\vskip 1cm


%%%%%%%%%%%%%%%%
\chapter*{Match Fixing}
%%%%%%%%%%%%%%%%
\addcontentsline{toc}{chapter}{Match Fixing}

\begin{center}

\textbf{Characters}

\vskip 1cm

\textbf{Jainam Sonetha}, Cricketer \\
\textbf{Bhargav Nandaniya}, Match Fixer \\
\vskip 1cm
\end{center}

\begin{description}[itemsep=1ex,leftmargin=1cm]
\setlength{\parskip}{5pt}

\item[] \hfill \\

\textit{Jainam is practising cricket as usual and Bhargav comes into the picture.}

\item[BHARGAV] Sir, kya me aapse do min baat kar sakta hu?

\item[JAINAM] \textit{(confused)} Aap? Aap yaha kyu aaye ho?

\item[BHARGAV] Kuch baatein karni thi! Apke matlab ki.

\item[JAINAM] \textit{(astonished)} Mere matlb ki baat aur aap? Mujhe apse koi baat nhi karni, mei janta hu aap match fixing krte ho and ese cricketers ko kharidte ho! But muje nahi kharid paoge!

\item[BHARGAV] Meri baat sunoge to khud bologe k, ha, I am ready to do so! Kabhi socha h itne matches mei acha perform kia to bhi kya mila! \textit{(pointing towards him)} Us player ko dekho, performance na hone k bawajud, uske pass kya kyaa hai!

\item[JAINAM] \textit{(angrily)} Aap kehna kya chahte ho?

\item[BHARGAV] Bas itna jese uss player k pass hai. Ye lambii gaddi, nauker chakarr, bada ghar, tumhare pass bhi sb kuch ho skta hai.

\textit{Starting the formation of boat by others.}

\item[BHARGAV] \textit{(continue)} Tum apni wife ko world tour karwa skte ho.

\textit{Harsh and Riya, mimicking \textbf{The Titanic} pose. }

\item[JAINAM] \textit{(momentarily becomes happy then comes out of his trance)} Nahi nahi, muje nhi chahye ye sb!

\item[BHARGAV] Areh suno to! Bhot sari ladkiya tumhari diwani hogi. Full night party-masti friends k sath.

\textit{other members will dance in the right corner of the stage}

\item[OTHERS] Party all night, party all night.

\item[JAINAM] \textit{gets excited and starts dancing with Neha}

\item[OTHERS] \textit{(pointing to Neha and Jainam)}

\item[BHARGAV] \textit{claps thrice} Soch k dekho jara. Dolat hogi shaurat hogi, to ab apka kya kehna?

\item[JAINAM] \textit{(joyous)} Oh Yes! I am ready.

\item[NEHA] Paise ne kharid liya hai insaan ko, Dar hai kahi kharid na le bhagwan ko!

\end{description}

\vskip 1cm


%%%%%%%%%%%%%%%%
\chapter*{Politics}
%%%%%%%%%%%%%%%%
\addcontentsline{toc}{chapter}{Politics}

\begin{center}

\textbf{Characters}

\vskip 1cm

\textbf{Kavita Jindal}, Politician \\
\textbf{Harsh Vakharia}, Bodyguard \& Chamcha \\
\textbf{Vijaan Masani}, Bodyguard \& Chamcha \\
\textbf{Neha Jain}, Reporter \\
\vskip 1cm
\end{center}

\begin{description}[itemsep=1ex,leftmargin=1cm]
\setlength{\parskip}{5pt}

\item[] \hfill \\

\textit{Kavita as aspiring politician is gathering citizens for vote appeal}

\item[HARSH] \textit{(loudly)} Hamara neta kesa ho?

\item[ALL] \textit{(loudly)} Munni devi jaisa ho.

\item[VIJAAN] \textit{(loudly)} Aree bhai kesa ho bhai kesa ho?

\item[ALL] \textit{(loudly)} Munni devi jaisa ho.

\item[KAVITA] \textit{(angrily)} Shh. Ye sab kya hai?

\item[HARSH \& VIJAAN] \textit{(jump at once)} Kya hua maidam!

\item[KAVITA] \textit{(confused)} Bas itni hi public? Maine to ktine paise diye the tum logo ko! Aur bas itne hi janta ayi h.

\item[HARSH] \textit{(jumps)} Are maidam, itnee paise me, itnaa hi milega!

\item[KAVITA] \textit{(angrily)} Mujhe to lagta hai tum logo ne hi kafii paise kha liye hai, aab jao piche!

\item[KAVITA] \textit{(continues)} Meri janta, vote deke muje apki seva krne ka maukka do, kuch krna chahti hu ap logo k liye!  Muje neta bana k bas fir dekho chamatkar.

\item[KAVITA] \textit{(again, angrily)} Ahh..

\item[HARSH \& VIJAAN] \textit{(jump at once)} Ab kya hua maidam?

\item[KAVITA] \textit{(again, angrily)} Kesi audience laye ho? Koi taali bhi nhi bajaa rha, reporter dekhenge to kya sochenge!

\item[HARSH \& VIJAAN] \textit{(at once)} Oye, taali bajaa oye.

\item[KAVITA] Mere ane se sab kuch milega. Gareeb ko milega..

\textit{Harsh and Vijaan acts}

\item[OTHERS] Khhaana.

\item[KAVITA] Piine ko milega..

\textit{Harsh and Vijaan acts}

\item[OTHERS] Paani.

\item[KAVITA] Pehenne ko milenge..

\textit{Harsh and Vijaan acts}

\item[OTHERS] Kaapde.

\item[HARSH \& VIJAAN] aur, \textit{(acts)} bijlii.

\item[HARSH] \textit{(loudly)} Hamara neta kesa ho?

\item[ALL] \textit{(loudly)} Munni devi jaisa ho.

\item[VIJAAN] \textit{(loudly)} Aree bhai kesa ho bhai kesa ho?

\item[ALL] \textit{(loudly)} Munni devi jaisa ho.

\item[VIJAAN] \textit{(humorously)} Munni devi jaisa ho.

\item[RIYA] Paiso ka rang, chadh gya hai inke sang, Ab dekhna kese badalta hai in inke rang dhang.

\item[NEHA] \textit{(waiting)} Kaha reh gaye neeta ji.

\item[HARSH] \textit{(loudly)} Munni devi..

\item[ALL] \textit{(loudly)} Zindabaad..

\item[VIJAAN] \textit{(loudly)} Munni devi..

\item[ALL] \textit{(loudly)} Zindabaad..

\item[NEHA] Maam, Aap itne bade maargin se jeete ho, to aap iska shrey kisse dena chahenge?

\item[KAVITA, HARSH \& VIJAAN] \textit{(acts)} Paaaiiissa..

\item[NEHA] Kya?

\item[KAVITA] Kuch nahi

\item[NEHA] Hume ye bhi pata chala hai ki central government apko 500 crore ka fund de rahi hai jaanta ki aabadi k liye.

\textit{(Kavita, Harsh and Vijaan dance)}

\item[KAVITA, HARSH \& VIJAAN] \textit{(sings)} Dene wala jabb bhi deta, deta chappad faad k.. gilli gilli appa.. gilli gilli appa.. Deta chappad faad ke!

\item[KAVITA] Desh aabad ho na ho, hum to aabad ho gye.

\item[VIJAAN] Inn murkho ko kaun bataye ki humne to votes kharid rakhe the.

\textit{(Kavita, Harsh and Vijaan freeze)}

\item[JAINAM] Bhrastachar ne bigaad diya hai iss desh ko, Kab badlega manav apne iss vesh ko.

\end{description}

\end{document}